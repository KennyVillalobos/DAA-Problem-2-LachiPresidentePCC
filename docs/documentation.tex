\documentclass[article]{llncs}
%
\usepackage[utf8]{inputenc}
\usepackage[spanish]{babel}
\usepackage{graphicx}
% Used for displaying a sample figure. If possible, figure files should
% be included in EPS format.
%
% If you use the hyperref package, please uncomment the following line
% to display URLs in blue roman font according to Springer's eBook style:
% \renewcommand\UrlFont{\color{blue}\rmfamily}


\begin{document}
%
\title{Dise\~{n}o y An\'alisis de Algoritmos. Problema 2: El Zool\'ogico}
%
%\titlerunning{Abbreviated paper title}
% If the paper title is too long for the running head, you can set
% an abbreviated paper title here
%
\author{Jes\'us Santos Capote y Kenny Villalobos Morales}
%
\institute{Facultad de Matemática y Computación, Universidad de La Habana, La Habana, Cuba }
%
\maketitle              % typeset the header of the contribution
%
\section{Definici\'on del Problema}

Se tiene un grafo no dirigido de $n$ vértices, unidos por arcos ponderados. Se desea computar para cada par 
de vértices $s$ y $t$ cuántos arcos pertenecen a algún camino de costo mínimo entre $s$ y $t$. La 
representación computacional del grafo de entrada será una matriz de costos.

\section{Primera Aproximaci\'on}

\subsection{Idea del Algoritmo}

Se le realiza al algoritmo de Dijkstra la siguiente modificaci\'on: Sea $d[u]$ el costo de llegar desde origen hasta el v\'ertice $u$.   
En cada nodo, en vez de guardar un padre, se guarda una lista de padres, de forma tal que  
cuando se haga $relax(u, v)$, si el costo 
computado hasta el momento de llegar a $v$  desde el origen $d[v]$, es igual al costo de ir del origen a $u$ $d[u]$ m\'as 
el costo del arco $(u,v)$ entonces a\~{n}adimos a la lista de padres de $v$ el v\'ertice $u$. Mediante estas listas de padres
podemos reconstruir todos los caminos de costo m\'inimo que llegan a $u$. Se tendr\'a una matriz \textbf{$solution$}
donde en la posici\'on $(i,j)$ almacenar\'a el conteo de la cantidad de caminos de costo m\'inimo entre el nodo $i$ y el nodo $j$. Por cada 
v\'ertice del grafo se efect\'ua el algoritmo de Dijkstra con la modificaci\'on anteriormente explicada. 
Cada vez que se termine una ejecuci\'on de Dijkstra para el origen de turno, llamemosle $u$, para cada par $(u, v)$ se recorren todos los caminos 
de costo m\'inimo encontrados y se cuentan todas los arcos que participan en dichos caminos, los arcos se 
contaran solo una vez, pues a medida que se recorren los caminos se ir\'a marcando en una matriz booleana los arcos ya contados. Por cada arco no marcado 
que se visite se aumenta en 1 el valor de \textbf{$solution[u, v]$}. 
Luego de esto se tendr\'an 
computados la cantidad de aristas que participan en caminos de costo m\'inimo para los pares donde 
origen de los caminos sea $u$, pares $(u,v)$, donde v es un v\'ertice del grafo distinto de $u$. Repitiendo este proceso para cada uno de los v\'ertices del grafo 
en \textbf{$solution$} estar\'a almacenada la respuesta deseada.

\subsection{Correctitud}

La modificaci\'on realizada a la operaci\'on relax que efect\'ua Dijkstra no afecta la correctitud 
del algoritmo, pues no se modifica su comportamiento, solo agregamos informaci\'on a los nodos. Luego 
podemos afirmar que se calculan correctamente los caminos de costo m\'inimo desde el or\'igen de turno 
hacia el resto de los nodos. Luego, el algoritmo calcula todos las caminos de costo m\'inimo entre cada 
par de nodos y cuenta los arcos que intervienen en los caminos una sola vez. 

\subsection{Complejidad Temporal}

La modificaci\'on hecha al algoritmo de Dijkstra no afecta su complejidad temporal, solo se agregan operaciones de complejidad 
constante a la operaci\'on relax.
Dijkstra tiene complejidad $O(|E| + |V|log|V|)$. Luego reconstruir todos los caminos de costo m\'inimo 
entre un par de v\'ertices es $O(V)$. En cada iteraci\'on del algoritmo se ejecuta una vez el algoritmo 
de Dijkstra y luego por cada uno de los $|V| - 1$ pares a analizar en la iteraci\'on se reconstruyen sus 
caminos de costo m\'inimo, luego pueden existir como m\'aximo $|E|$ camino de costo m\'inimo diferentes 
de un origen a un destino, luego el costo de contar todos los caminos de costo m\'inimo entre los $|V| - 1$
pares es $O((|V|-1)|E||V|)$ = $O(|V|^2|E|)$ = $O(|V|^4)$. Luego el coste de una iteraci\'on es $O(|E| + |V|log|V| + |V|^4)$. 
El algoritmo realiza $|V|$ iteraciones. Por tanto la complejidad temporal del algoritmo es $O(|V|*(|E| + |V|log|V| + |V|^4))$, 
que es $O(|V|^5)$

\end{document}