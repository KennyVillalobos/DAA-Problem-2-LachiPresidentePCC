\documentclass[article]{llncs}
%
\usepackage[utf8]{inputenc}
\usepackage[spanish]{babel}
\usepackage{graphicx}
% Used for displaying a sample figure. If possible, figure files should
% be included in EPS format.
%
% If you use the hyperref package, please uncomment the following line
% to display URLs in blue roman font according to Springer's eBook style:
% \renewcommand\UrlFont{\color{blue}\rmfamily}


\begin{document}
%
\title{Dise\~{n}o y An\'alisis de Algoritmos. Problema 2: El Zool\'ogico}
%
%\titlerunning{Abbreviated paper title}
% If the paper title is too long for the running head, you can set
% an abbreviated paper title here
%
\author{Jes\'us Santos Capote y Kenny Villalobos Morales}
%
\institute{Facultad de Matemática y Computación, Universidad de La Habana, La Habana, Cuba }
%
\maketitle              % typeset the header of the contribution
%
\section{Definici\'on del Problema}

Se tiene un grafo no dirigido de $n$ vértices, unidos por arcos ponderados. Se desea computar para cada par 
de vértices $s$ y $t$ cuántos arcos pertenecen a algún camino de costo mínimo entre $s$ y $t$. La 
representación computacional del grafo de entrada será una matriz de costos.

\section{Primera Aproximaci\'on}

\subsection{Idea del Algoritmo}

Se le realiza al algoritmo de Dijkstra la siguiente modificaci\'on: Sea $d[u]$ el costo de llegar desde origen hasta el v\'ertice $u$.   
En cada nodo, en vez de guardar un padre, se guarda una lista de padres, de forma tal que  
cuando se haga $relax(u, v)$, si el costo 
computado hasta el momento de llegar a $v$  desde el origen $d[v]$, es igual al costo de ir del origen a $u$ $d[u]$ m\'as 
el costo del arco $(u,v)$ entonces a\~{n}adimos a la lista de padres de $v$ el v\'ertice $u$. Mediante estas listas de padres
podemos reconstruir todos los caminos de costo m\'inimo que llegan a $u$. Se tendr\'a una matriz \textbf{$solution$}
donde en la posici\'on $(i,j)$ almacenar\'a el conteo de la cantidad de caminos de costo m\'inimo entre el nodo $i$ y el nodo $j$. Por cada 
v\'ertice del grafo se efect\'ua el algoritmo de Dijkstra con la modificaci\'on anteriormente explicada. 
Cada vez que se termine una ejecuci\'on de Dijkstra para el origen de turno, llamemosle $u$, para cada par $(u, v)$ se recorren todos los caminos 
de costo m\'inimo encontrados y se cuentan todas los arcos que participan en dichos caminos, los arcos se 
contaran solo una vez, pues a medida que se recorren los caminos se ir\'a marcando en una matriz booleana los arcos ya contados. Por cada arco no marcado 
que se visite se aumenta en 1 el valor de \textbf{$solution[u, v]$}. 
Luego de esto se tendr\'an 
computados la cantidad de aristas que participan en caminos de costo m\'inimo para los pares donde 
origen de los caminos sea $u$, pares $(u,v)$, donde v es un v\'ertice del grafo distinto de $u$. Repitiendo este proceso para cada uno de los v\'ertices del grafo 
en \textbf{$solution$} estar\'a almacenada la respuesta deseada.

\subsection{Correctitud}

La modificaci\'on realizada a la operaci\'on relax que efect\'ua Dijkstra no afecta la correctitud 
del algoritmo, pues no se modifica su comportamiento, solo agregamos informaci\'on a los nodos. Luego 
podemos afirmar que se calculan correctamente los caminos de costo m\'inimo desde el or\'igen de turno 
hacia el resto de los nodos. Luego, el algoritmo calcula todos las caminos de costo m\'inimo entre cada 
par de nodos y cuenta los arcos que intervienen en los caminos una sola vez. 

\subsection{Complejidad Temporal}

La modificaci\'on hecha al algoritmo de Dijkstra no afecta su complejidad temporal, solo se agregan operaciones de complejidad 
constante a la operaci\'on relax.
La modalidad de Dijkstra utilizada tiene complejidad $O(|V|^2)$. Luego reconstruir todos los caminos de costo m\'inimo 
entre un par de v\'ertices es $O(V)$. En cada iteraci\'on del algoritmo se ejecuta una vez el algoritmo 
de Dijkstra y luego por cada uno de los $|V| - 1$ pares a analizar en la iteraci\'on se reconstruyen sus 
caminos de costo m\'inimo, luego pueden existir como m\'aximo $|E|$ camino de costo m\'inimo diferentes 
de un origen a un destino, luego el costo de contar todos los caminos de costo m\'inimo entre los $|V| - 1$
pares es $O((|V|-1)|E||V|)$ = $O(|V|^2|E|)$ = $O(|V|^4)$. Luego el coste de una iteraci\'on es $O(|E| + |V|log|V| + |V|^4)$. 
El algoritmo realiza $|V|$ iteraciones. Por tanto la complejidad temporal del algoritmo es $O(|V|*(|E| + |V|log|V| + |V|^4))$, 
que es $O(|V|^5)$

\section{Segunda Solución}

\subsection{Idea del algoritmo}

Realizamos el algoritmo de Floyd-Warshall a partir de la matriz de costos inicial (cost) para obtener la matriz de las distancias de costo mínimo entre todo par de vértices $(i,j)$, llamemos a esta matriz $fw$.
Luego, para cada par de vértices $(v_i,v_j)$, con $i \neq j$, hallamos todos los vértices $v_k$, tal que $fw[i,j] = fw[i,k] + fw[k,j]$, estos vértices pertencen a algún
camino de costo mínimo de $i$ a $j$, al conjunto de estos vértices para cada par $(i,j)$ llamémoslo $CCM_(i,j)$. Luego, para cada par de vértices $(v_i,v_j)$, con $i \neq j$, hallamos el número de vértices $v_p$ que cumplen que:
$fw[i,p] + dist[p,j] = fw[i,j]$, esta es la cantidad de aristas que llegan a $j$, pertenecientes a algún camino de costo mínimo proveniente de $i$, llamemos a este número $e_(i,j)$.
Luego hacemos un recorrido por cada par de vértices $(v_i,v_j)$ la cantidad de aristas que pertenecen a algún camino de costo mínimo de $i$ a $j$ es igual a la suma de las aristas que llegan a cada vértice $k$ perteneciente a $CCM_(i,j)$ en caminos de costo
mínimo provenientes de $i$:

$S_(v_i,v_j) = \sum e_(i,k), \forall k\in CCM_(i,j) $

\subsection{Correctitud}
Primero demostremos que si una arista pertenece a algún camino de costo mínimo entre $i$ y $j$, el algoritmo la cuenta:

Si una arista $(x,y)$ pertence a algún camino de costo mínimo entre $i$ y $j$, entonces: $y$ es un vértice que pertence a algún camino de costo mínimo entre $i$ y $j$ y se cumple que: 
el costo mínimo de un camino de $i$ a $y$ sumado con el costo de un camino de costo mínimo de $y$ a $j$ es igual al costo de algún camino de costo mínimo de $i$ a $j$

Luego, si $(x,y)$ pertenece a algún camino de costo mínimo entre $i$ y $j$, $x,y$ pertenece a algún camino de costo mínimo entre $i$ y $y$, pues todo subcamino de un camino de costo mínimo es de costo mínimo.
Finalmente si $(x,y)$ pertenece a algún camino de costo mínimo entre $i$ y $y$, entonces se cumple que la suma del costo de la arista $(x,y)$ con el costo de un camino de costo mínimo de $i$ a $x$ debe ser igual
al costo de un camino de costo mínimo de $i$ a $y$. 

Por lo cual: $ y \in CCM_(i,j) \land (x,y) \in e_(i,y)$, y por lo tanto el algoritmo cuenta a $(x,y)$

Ahora demostremos que si el algoritmo cuenta una arista, esta pertence a algún camino de costo mínimo ente $i$ y $j$:


\subsection{Complejidad Temporal}

El algoritmo genera la matriz de los costos caminos de costo mínimo para todos los pares de vértices $(v_i,v_j)$ utilizando el algoritmo
de Floyd-Warshall, este tiene una complejidad temporal $O(|V|^3)$, luego por cada elemento de la matriz, o sea, par de vértices, (excepto los pares $(i,j)$, con $i=j$)
el algoritmo recorre todos los vértices del grafo (excepto $i$), para reconocer cuales vértices pertenecen a caminos de costo mínimo entre $i$ y $j$ y cuántas aristas de la forma $(k,j)$ pertenecen
 a caminos de costo mínimo de $i$ a $j$, esta comprobación sucede en tiempo constante $c$, por lo cual este procedimiento tiene una complejidad temporal $O(|V|^3)$. Luego, para cada par de vértices $(i,j)$ se recorren los vértices previamente calculados
 que pertenecen a caminos de costo mínimo de $i$ a $j$ (a lo sumo $|V|-1$) y se suma el número de aristas que llegan a cada uno de esos vértices pertenecientes a caminos de costo mínimo provenientes de $i$, este procedimiento tiene una complejidad temporal $O(|V|^3)$.
 Por tanto la complejidad temporal del algoritmo es $O(3*|V|^3)$, que es $O(|V|^3)$.


\end{document}